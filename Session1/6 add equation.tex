\documentclass[12pt, letterpaper]{article}
\begin{document}
if you want to print something in line \(E=mc^2\), but not like \{A=B\}. 
The mass-energy equivalence is described by the famous equation
\[ E=mc^2 \] discovered in 1905 by Albert Einstein. 
\[ 
F = G \left( \frac{m_1 m_2}{r^2} \right)
\]


\[ 
 \left[  \frac{ N } { \left( \frac{L}{p} \right)  - (m+n) }  \right]
\]




In natural units ($c = 1$), the formula expresses the identity. same method as figure but this is a equation \ref{eq:thiseq}.
\begin{equation}
E= \frac{a^b}{c_{d \times e} } 
\label{eq:thiseq}
\end{equation}

\begin{equation}
    \label{eq:1}
    U= v \cup c
\end{equation}

\begin{equation}
    \label{eq:3}
    U= \left[ \frac{c}{ \frac{L}{p}} \right]  
\end{equation}




\[ \int\limits_0^1 x^2 + y^2 \ dx \]



and notice the difference \[ \int_0^1 x^2 + y^2 \ dx \]
\end{document}

More on brackets
%https://www.overleaf.com/learn/latex/Brackets_and_Parentheses
%https://www.overleaf.com/learn/latex/Subscripts_and_superscripts