\documentclass[12pt, a4paper]{article}
\title{My first LaTeX document}
\author{Adnan Ali \thanks{Corresponding author.}}
\date{December 2023}

 



\begin{document}
\maketitle
\section{Introduction}
We have now added a title, author and date to our first \LaTeX{} document!
This line here is a comment. It will not be typeset in the document.
We have now added a title, author and date to our first \LaTeX{} document!
This line here is a comment. It will not be typeset in the document.
We have now added a title, author and date to our first \LaTeX{} document!
This line here is a comment. It will not be typeset in the document.
\par
\noindent We have now added a title, author and date to our first \LaTeX{} document!
This line here is a comment. It will not be typeset in the document.
We have now added a title, author and date to our first \LaTeX{} document!
This line here is a comment. It will not be typeset in the document.
We have now added a title, author and date to our first \LaTeX{} document!
This line here is a comment. It will not be typeset in the document.

\section{Bold italic}
if i want to \textbf{bold} \textbf{new bold}, \textbf{bold}, \textit{italic}, \textit{abc}, \emph{italic}

We have now added a title, author and date to our first \LaTeX{} document!
This line here is a comment. ``it will not be typeset in the document".



\section{Heading}
If i want to \textbf{bold}, \textbf{bold 2} or \textit{italic}, \textit{italic 2}, \emph{italic}, \underline{some text}.  



here we will be explaining how to \textbf{bold}, or \textbf{bold}, \emph{italic}, \textit{italic} or \textit{with keyboard italic} the text. 
Some of the \textbf{greatest}
discoveries in \underline{science} 
were made by \textbf{\textit{accident}}.


Another very useful command is \emph{argument}, whose effect on its argument depends on the context. Inside normal text, the emphasized text is italicized, but this behaviour is reversed if used inside an italicized text—see the next example:

Some of the greatest \emph{discoveries} in science 
were made by accident.
\textit{some text and it is not \emph{italic} }
\textit{Some of the greatest \emph{discoveries} 
in science were made by accident.}

\textbf{Some of the greatest \emph{discoveries} 
in science were made by accident.}


\section{Results}
Another very useful command is \emph{argument}, whose effect on its argument depends on the context. Inside normal text, the emphasized text is italicized, but this behaviour is reversed if used inside an italicized text—see the next example. 
\par

\noindent Another very useful command is \emph{argument}, whose effect on its argument depends on the context. Inside normal text, the emphasized text is italicized, but this behaviour is reversed if used inside an italicized text—see the next example.

this is line forty two. 



\end{document}